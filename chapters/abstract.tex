% -------------------------------------------------------
%  Proposal Abstract at Most 250 words
% -------------------------------------------------------
چکیده شما در این بخش قرار میگیرد. تعریفی از چکیده بر اساس \hyperref{https://fa.wikipedia.org/wiki/%DA%86%DA%A9%DB%8C%D8%AF%D9%87_(%DA%A9%D8%AA%D8%A7%D8%A8%D8%AF%D8%A7%D8%B1%DB%8C)}{category}{name}{ویکیدیا}:
	چکیده خلاصه‌ای از یک مقاله تحقیقاتی، پایان‌نامه، بررسی، همایش دانشگاهی یا هر گونه تجزیه و تحلیل موضوع خاص است؛ و اغلب استفاده می‌شود که به خواننده کمک کند تا به سرعت، هدف مقاله را دریابد؛ چکیده همیشه در ابتدای یک دست نوشته یا یک نسخه تایپ شده ظاهر می‌شود و به عنوان نقطه ورود برای هر مقاله علمی یا درخواست حق ثبت اختراع می‌باشد. چکیده و نمایه سازی خدماتی برای رشته‌های مختلف علمی، در تدوین بدنه‌ای از ادبیات برای آن موضوع خاص فراهم می‌کند.
	چکیده یا خلاصه شرایط در برخی از نشریات استفاده می‌شود. اما باید اذعان کرد که در گزارش‌های مدیریتی، یک چکیده اجرائی معمولاً اطلاعات بیشتری ( و اغلب اطلاعات حساس تری) را نسبت به چکیده دربر می‌گیرد.
	یک چکیده ممکن است به عنوان یک نهاد مستقل به جای یک مقاله کامل عمل کند. به عنوان مثال یک چکیده توسط بسیاری از سازمان‌ها به عنوان مبنایی برای انتخاب پژوهش استفاده می‌شود که در آن صورت هدف او برای ارائه در شکل‌هایی از پوستر، پلت فرم یا ارائه کارگاه در یک کنفرانس علمی می‌باشد.
	بیشتر موتورهای جستجو که پایگاه‌های ادبیات را جستجو می‌کنند چکیده‌ها را بیشتر از متن کامل مقالات فراهم می‌کنند. متون کامل مقالات علمی اغلب باید به دلیل کپی رایت یا هزینه‌های ناشر خریداری شود و در نتیجه می‌توان گفت که چکیده یک نقطه فروش قابل توجه برای چاپ یا فرم‌های الکترونیکی متن کامل می‌باشد.